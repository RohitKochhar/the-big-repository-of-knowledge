\documentclass{article}
\usepackage[utf8]{inputenc}
\usepackage{fullpage}
\usepackage{verbatim}
\usepackage{graphicx}
\usepackage{float}

\title{InfluxDB}
\author{Rohit Singh}
\date{June 26th - July 2nd, 2022}

\begin{document}

\maketitle

\tableofcontents

\section{Overview}

\subsection{What is InfluxDB}

\subsubsection{Influxdata for timeseries development}

InfluxDB is created by \textbf{influxdata}, who also create \textbf{Telegraf} and \textbf{Flux}. These threee products create a hollistic development platform, used for IoT applications, financial applications and more. InfluxDB is the leading timeseries database engine. Flux is a functional scripting and querying language and Telegraf is the leading metrics collector for timeseries data. 

Using these three products, we can store timeseries data in a InfluxDB database, query data with Flux and aggregate metrics with Telegraf.

\section{Getting Started}\label{ref:getting-started}

For the purposes of this chapter, we will use Docker and Docker Compose to create local InfluxDB instances.

\subsection{Docker}

If we just need a quick instance up and running, we don't need to use the preconfigured \verb|docker-compose.yaml| file, and we can instead run using Docker on the command-line, we use the following command:

\begin{verbatim}
    $ docker run -p 8086:8086 \
      -v $PWD:/var/lib/influxdb2 \
      influxdb:2.0
\end{verbatim}

Note that this will prompt us for a initial user, password and storage information every time, whereas with the docker compose creation, as explained in section \ref{docker-compose}, our configuration is saved as IaC (infrastructure as code), allowing for a more easily repeatable and consistent creation process.

\subsection{Docker Compose}\label{docker-compose}

The sample image creation using Docker Compose is shown in \verb|./InfluxDB/docker-compose.yaml|. Although the file is subject to variation, on a high-level it looks like this:

\begin{verbatim}
version: "3.9"

services:
  influxdb:
    image: influxdb:latest
    ports:
      - "8086:8086"
    volumes:
      - ./influxdb2:/var/lib/influxdb2:rw
    environment:
      - DOCKER_INFLUXDB_INIT_USERNAME=root
      - DOCKER_INFLUXDB_INIT_PASSWORD=rootpassword
      - DOCKER_INFLUXDB_INIT_ORG=init_org
      - DOCKER_INFLUXDB_INIT_BUCKET=init_bucket
      - INFLUXDB_HTTP_AUTH_ENABLED="true"
\end{verbatim}

The key takeaways are that we expose the service on port 8086, which lets us view the dashboard by navigating to \verb|localhost:8086|. We create a shared volume between our local machine and the image, located in the directory where the \verb|docker-compose up| command is run. 

We initialize the database with a simple pair of root user credentials, and we create an initial bucket and organization (described in sections \ref{ref:bucket} and \ref{ref:organization}, respectively)

Before running any programs that depend on InfluxDB, ensure that you've run either the docker-compose or docker command and can log in at the UI via \verb|localhost:8086|.

\subsection{API Keys}

To authorize clients, we need API keys. To retreive an API key, go to \verb|localhost:8086| and log-in. Once logged in, from the navigation tool bar > data explorer > API Keys. Copy the resulting key and save it as an environment variable (recommended) or in the file as a string.

While it is recommended to not expose secrets in code files, we make the exception here for simplicity.

\section{Tutorials}

In this section, we go through some ways to access the influx database using its Python client.

\subsection{Simple Write}

The source code for this tutorial can be found at \\ \verb|the-big-repository-of-knowledge/InfluxDB/tutorials/simpleWrite.py|

\subsubsection{Introduction}

In this tutorial, we are creating a simple program to simulate a temperature reading and write the result to our InfluxDB instance located at \verb|localhost:8086| (this assumes you have created an instance as outlined in section \ref{ref:getting-started})

\section{Dictionary}

\subsection{Buckets}\label{ref:bucket}

\textbf{Buckets} are named locations where the time series data is stored.

Buckets have an associated \textit{retention period} which specifies the duration that each data point should persist. Once a data points timestamp becomes older than the retention period, InfluxDB drops the point.

Each bucket belongs to an \textit{organization}, which is discussed in section \ref{ref:organization}

\subsection{Organization}\label{ref:organization}

\textbf{Organizations} are workspaces for groups of users. All dashboards, tasks, buckets, members, etc., belong to an organization


\end{document}